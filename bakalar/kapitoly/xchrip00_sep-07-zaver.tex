\chapter{Záver}
V~rámci práce boli zhromaždené a naštudované informáciu potrebné pre návrh a implementáciu testov, pričom návrh testov je súčasťou tejto práce.
Potrebné informácie sú rozdelené na: požiadavky na testy systému, prehľad modelov životného cyklu vývoja softvéru, teórie testovania, princíp systému Fitcrack a návrhu samotných testov.
Takisto bola overená funkčnosť architektúry akceptačných testov pomocou skriptu, ktorý dokáže čítať a interpretovať údaje z~databázy.
Problém by mohol byť s~posielaním výsledkov testovania runner-u na klientovi, no spolieham sa na BOINC a funkcionalitu.

Ďalšia práca by sa mala zamerať na meranie pokrytia kódu/logiky systému testovacími prípadmi a samotnú iplementáciu testovacích prípadov, pre oba spôsoby testovania.
Je potrebné stanoviť metriky a spôsoby merania pokrytia logiky systému testovacími prípadmi.
Pre implementáciu jednotkových testov je vítaná úzka spolupráca s~vývojármi konkrétnych modulov, no nie je vyžadovaná.
Kapitola o~testovaní softvéru~\ref{testing} obsahuje veľké množstvo pojmov, ktoré nie sú formálne definované, preto by bolo vhodné ešte pred samotnou implementáciou testov naštudovať a doplniť tieto definície.

Testy systému by mohli byť použité aj na diagnostiku softvéru v~prevádzke, no to bude vyžadovať spoluprácu s~vývojármi WebAdmin-a a REST API, ktoré komunikuje so serverovými modulmi, aby výsledky testov/ diagnostiky mohli byť vizualizované užívateľovi.
